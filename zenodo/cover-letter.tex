\documentclass[11pt]{letter}
\usepackage[a4paper,margin=1in]{geometry}
\usepackage{hyperref}
\hypersetup{hidelinks}

% --- Custom commands for cleaner layout ---
\usepackage{lmodern} % Recommended font for better looks
\usepackage{parskip} % To control paragraph spacing

\name{Alexandre Furtado Neto}
\address{UNESP Alumnus\\
Email: \href{mailto:alexandre.com@yahoo.com}{alexandre.com@yahoo.com}\\
ORCID: \href{https://orcid.org/0000-0001-9435-6566}{0000-0001-9435-6566}}

\date{\today}

\begin{document}

\begin{letter}{Professor Eleanor Knox, Editor-in-Chief\\
\textit{Foundations of Physics}}

\opening{Dear Professor Knox,}

I am writing to submit the manuscript entitled \textbf{"It from bit: a concrete attempt"} for consideration as a research paper in \textit{Foundations of Physics}. This paper proposes a toy-universe framework on a discrete three-torus with an auxiliary dimension, where binary strings evolve by simple deterministic rules. These rules generate wavefronts, harmonic behavior, and polarization, which are subsequently mapped to particle-like structures, electromagnetic interactions, and effective gravitational features.

\section*{Novelty}
Our model introduces three key novelties: (i) a convolution-like axis that enables non-local transfer without probabilistic superposition; (ii) an explicit six-bit charge encoding that reflects conservation principles; and (iii) an affinity mechanism that binds localized excitations into aggregates exhibiting self-interference and entanglement-like correlations.

\section*{Fit}
By combining logic, arithmetic, and topology into a deterministic cellular-automaton substrate, this work directly addresses foundational questions in quantum mechanics and emergent relativity. Its clear falsifiability (e.g., electron self-interference trace tests) makes it particularly appropriate for the conceptual scope of \textit{Foundations of Physics}.

\section*{Significance}
The manuscript provides detailed specifications for state variables and evolution rules, outlines heuristic constructions of fermions, bosons, and gauge mediators, and identifies specific experimental scenarios that could validate or refute the proposal.

I confirm that I am the sole author of this work, received no external funding for its completion, and have no competing interests to declare.

Thank you for considering this submission. I am confident this work will be of significant interest to your readership, particularly those engaged with computational and conceptual approaches to fundamental physics.

\closing{Sincerely,}

\signature{Alexandre Furtado Neto}

\end{letter}
\end{document}